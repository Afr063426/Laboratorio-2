% Options for packages loaded elsewhere
\PassOptionsToPackage{unicode}{hyperref}
\PassOptionsToPackage{hyphens}{url}
%
\documentclass[
  oneside]{memoir}
\usepackage{amsmath,amssymb}
\usepackage{lmodern}
\usepackage{ifxetex,ifluatex}
\ifnum 0\ifxetex 1\fi\ifluatex 1\fi=0 % if pdftex
  \usepackage[T1]{fontenc}
  \usepackage[utf8]{inputenc}
  \usepackage{textcomp} % provide euro and other symbols
\else % if luatex or xetex
  \usepackage{unicode-math}
  \defaultfontfeatures{Scale=MatchLowercase}
  \defaultfontfeatures[\rmfamily]{Ligatures=TeX,Scale=1}
\fi
% Use upquote if available, for straight quotes in verbatim environments
\IfFileExists{upquote.sty}{\usepackage{upquote}}{}
\IfFileExists{microtype.sty}{% use microtype if available
  \usepackage[]{microtype}
  \UseMicrotypeSet[protrusion]{basicmath} % disable protrusion for tt fonts
}{}
\makeatletter
\@ifundefined{KOMAClassName}{% if non-KOMA class
  \IfFileExists{parskip.sty}{%
    \usepackage{parskip}
  }{% else
    \setlength{\parindent}{0pt}
    \setlength{\parskip}{6pt plus 2pt minus 1pt}}
}{% if KOMA class
  \KOMAoptions{parskip=half}}
\makeatother
\usepackage{xcolor}
\IfFileExists{xurl.sty}{\usepackage{xurl}}{} % add URL line breaks if available
\IfFileExists{bookmark.sty}{\usepackage{bookmark}}{\usepackage{hyperref}}
\hypersetup{
  hidelinks,
  pdfcreator={LaTeX via pandoc}}
\urlstyle{same} % disable monospaced font for URLs
\usepackage{color}
\usepackage{fancyvrb}
\newcommand{\VerbBar}{|}
\newcommand{\VERB}{\Verb[commandchars=\\\{\}]}
\DefineVerbatimEnvironment{Highlighting}{Verbatim}{commandchars=\\\{\}}
% Add ',fontsize=\small' for more characters per line
\usepackage{framed}
\definecolor{shadecolor}{RGB}{248,248,248}
\newenvironment{Shaded}{\begin{snugshade}}{\end{snugshade}}
\newcommand{\AlertTok}[1]{\textcolor[rgb]{0.94,0.16,0.16}{#1}}
\newcommand{\AnnotationTok}[1]{\textcolor[rgb]{0.56,0.35,0.01}{\textbf{\textit{#1}}}}
\newcommand{\AttributeTok}[1]{\textcolor[rgb]{0.77,0.63,0.00}{#1}}
\newcommand{\BaseNTok}[1]{\textcolor[rgb]{0.00,0.00,0.81}{#1}}
\newcommand{\BuiltInTok}[1]{#1}
\newcommand{\CharTok}[1]{\textcolor[rgb]{0.31,0.60,0.02}{#1}}
\newcommand{\CommentTok}[1]{\textcolor[rgb]{0.56,0.35,0.01}{\textit{#1}}}
\newcommand{\CommentVarTok}[1]{\textcolor[rgb]{0.56,0.35,0.01}{\textbf{\textit{#1}}}}
\newcommand{\ConstantTok}[1]{\textcolor[rgb]{0.00,0.00,0.00}{#1}}
\newcommand{\ControlFlowTok}[1]{\textcolor[rgb]{0.13,0.29,0.53}{\textbf{#1}}}
\newcommand{\DataTypeTok}[1]{\textcolor[rgb]{0.13,0.29,0.53}{#1}}
\newcommand{\DecValTok}[1]{\textcolor[rgb]{0.00,0.00,0.81}{#1}}
\newcommand{\DocumentationTok}[1]{\textcolor[rgb]{0.56,0.35,0.01}{\textbf{\textit{#1}}}}
\newcommand{\ErrorTok}[1]{\textcolor[rgb]{0.64,0.00,0.00}{\textbf{#1}}}
\newcommand{\ExtensionTok}[1]{#1}
\newcommand{\FloatTok}[1]{\textcolor[rgb]{0.00,0.00,0.81}{#1}}
\newcommand{\FunctionTok}[1]{\textcolor[rgb]{0.00,0.00,0.00}{#1}}
\newcommand{\ImportTok}[1]{#1}
\newcommand{\InformationTok}[1]{\textcolor[rgb]{0.56,0.35,0.01}{\textbf{\textit{#1}}}}
\newcommand{\KeywordTok}[1]{\textcolor[rgb]{0.13,0.29,0.53}{\textbf{#1}}}
\newcommand{\NormalTok}[1]{#1}
\newcommand{\OperatorTok}[1]{\textcolor[rgb]{0.81,0.36,0.00}{\textbf{#1}}}
\newcommand{\OtherTok}[1]{\textcolor[rgb]{0.56,0.35,0.01}{#1}}
\newcommand{\PreprocessorTok}[1]{\textcolor[rgb]{0.56,0.35,0.01}{\textit{#1}}}
\newcommand{\RegionMarkerTok}[1]{#1}
\newcommand{\SpecialCharTok}[1]{\textcolor[rgb]{0.00,0.00,0.00}{#1}}
\newcommand{\SpecialStringTok}[1]{\textcolor[rgb]{0.31,0.60,0.02}{#1}}
\newcommand{\StringTok}[1]{\textcolor[rgb]{0.31,0.60,0.02}{#1}}
\newcommand{\VariableTok}[1]{\textcolor[rgb]{0.00,0.00,0.00}{#1}}
\newcommand{\VerbatimStringTok}[1]{\textcolor[rgb]{0.31,0.60,0.02}{#1}}
\newcommand{\WarningTok}[1]{\textcolor[rgb]{0.56,0.35,0.01}{\textbf{\textit{#1}}}}
\usepackage{graphicx}
\makeatletter
\def\maxwidth{\ifdim\Gin@nat@width>\linewidth\linewidth\else\Gin@nat@width\fi}
\def\maxheight{\ifdim\Gin@nat@height>\textheight\textheight\else\Gin@nat@height\fi}
\makeatother
% Scale images if necessary, so that they will not overflow the page
% margins by default, and it is still possible to overwrite the defaults
% using explicit options in \includegraphics[width, height, ...]{}
\setkeys{Gin}{width=\maxwidth,height=\maxheight,keepaspectratio}
% Set default figure placement to htbp
\makeatletter
\def\fps@figure{htbp}
\makeatother
\setlength{\emergencystretch}{3em} % prevent overfull lines
\providecommand{\tightlist}{%
  \setlength{\itemsep}{0pt}\setlength{\parskip}{0pt}}
\setcounter{secnumdepth}{-\maxdimen} % remove section numbering
\usepackage[left=15mm, inner=3cm, outer=3cm, top=3cm,bottom=3cm ]{geometry}
\usepackage{mathpazo}
\usepackage[spanish]{babel}
\usepackage[utf8]{inputenc}
\usepackage{graphics,graphicx}
\usepackage{wallpaper}
\usepackage{multicol}
\usepackage{multirow}
\usepackage{float}
\nonzeroparskip
\setlength{\parindent}{0pt}
%\usepackage{parskip}
%\setcounter{secnumdepth}{1}
\renewcommand\thesection{\arabic{section}}

\copypagestyle{headerwchap}{myheadings}
\chapterstyle{bianchi}
 \makeevenhead{headerwchap}{\thepage}{}{}
 \makeoddhead{headerwchap}{}{}{\thepage}
 \makeoddfoot{headerwchap}{}{}{}
 \makeevenfoot{headerwchap}{}{}{}
\makeheadrule{headerwchap}{\textwidth}{0.5pt}

\usepackage[ %
style=authoryear, % 
natbib = true, %
dashed = false, %
firstinits = true, 
url =false, %
isbn = false, %
backend = bibtex]{biblatex}
\addbibresource{referencias_proyecto.bib}
%%%%%%%%%%%%%%%%%%%%%%%%%%%%%%%%%%%

%%%%%%%%%%%%%%%%%%%%%%%%%%%%%%%%%%
%%%% EDITAR SOLO EL SEGUNDO %%%%%%%
%%%% CONJUNTO DE PARENTESIS %%%%%%%
%%%%%%%%%%%%%%%%%%%%%%%%%%%%%%%%%%%

% Titulo del proyecto
\newcommand{\TituloProy}{Segundo laboratorio}
 
% Nombre de Estudiantes y Carnet
\newcommand{\EstudianteUno}{Cervantes Artavia Joshua}
\newcommand{\EstudianteDos}{Sabater Guzmán Daniel}

\newcommand{\ds}[1]{\ensuremath{\displaystyle{#1}}}

\makeindex

\def\bitcoinB{\leavevmode
  {\setbox0=\hbox{\textsf{C}}%
    \dimen0\ht0 \advance\dimen0 0.2ex
    \ooalign{\hfil \box0\hfil\cr
      \hfil\vrule height \dimen0 depth.2ex\hfil\cr
    }%
  }%
}
\usepackage{booktabs}
\usepackage{longtable}
\usepackage{array}
\usepackage{multirow}
\usepackage{wrapfig}
\usepackage{float}
\usepackage{colortbl}
\usepackage{pdflscape}
\usepackage{tabu}
\usepackage{threeparttable}
\usepackage{threeparttablex}
\usepackage[normalem]{ulem}
\usepackage{makecell}
\usepackage{xcolor}
\ifluatex
  \usepackage{selnolig}  % disable illegal ligatures
\fi

\author{}
\date{\vspace{-2.5em}}

\begin{document}
\frontmatter

\begin{titlingpage}
  \newcommand{\HRule}{\rule{\linewidth}{0.5mm}} % Defines a new command for the horizontal lines, change thickness here

  \begin{center}

    {\Huge \textbf{UNIVERSIDAD DE COSTA RICA}}\\[0.2cm] 
    {\Large ESCUELA DE MATEM\'ATICA\\[0.2cm]
DEPARTAMENTO DE MATEMÁTICA PURA Y CIENCIAS ACTUARIALES\\
AN\'ALISIS DE INSTRUMENTOS DE INVERSI\'ON\\[0cm]
} %

   

    \HRule \\[0.4cm]
    \Large \textbf{\TituloProy}
    \HRule \\[0.4cm]

    \vfill
      
    {\Large{\textbf{Modelo de Nelson y Siegel}}}
   
    \vfill
    \begin{center}
      Realizado por %
      \linebreak %
      \linebreak %
      \begin{tabular}{c}
        \hline  \\
        Cervantes Artavia Joshua  \\ \\
        Sabater Guzmán Daniel  \\ \\
        \hline 
      \end{tabular}
    \end{center}

    \vfill
    

    

    % Bottom of the page
    \begin{center}
    \begin{tabular}{l @{\hskip 2in} r}
    \includegraphics[height=2cm]{./images/ucr_marca_de_agua}
  & \includegraphics[height=4cm]{./images/EMat_escuela_matematica_horizontal_1} 
    \end{tabular}
    \end{center}

 %\\[0.1in]
%     \Large{Escuela de Matemática}\\
%     \Large{\textsc{Universidad de Costa Rica}}\\


    \ThisURCornerWallPaper{0.5}{./images/granosgirasol.png}
 
  \end{center}

\end{titlingpage}

%%% Local Variables:
%%% mode: latex
%%% TeX-master: "plantilla_proyecto"
%%% End:

{
\setcounter{tocdepth}{1}
\tableofcontents
}
\mainmatter
\chapter{Resumen Ejecutivo 1}

Aquí debe dar resultados generales de lo que obtuvo del análisis.

\chapter{Introducción 1}

\chapter{Objetivos 1}

\section{Objetivo general}

Analizar la curva de rendimiento de los bonos del tesoro de Estados
Unidos de Norte América por medio del modelo de Nelson y Siegel en el
periodo de años comprendido entre el 1° de enero del 2010 y el 20 de
mayo del 2021.

\section{Objetivos específicos}
\begin{enumerate} 

\item Explicar el modelo de Nelson y Siegel.

\item Describir las características de la base de datos de los bonos del tesoro de Estados Unidos de Norte América.

\item Implementar el modelo de Nelson y Siegel en lenguaje de programación R, para la base de datos de los bonos del tesoro de Estados Unidos de Norte América.

\item Evaluar los resultados obtenidos apartir del modelo de Nelson y Siegel.

\end{enumerate}

\chapter{Marco teórico 3}
\par

La curva de rendimiento es una herramienta útil a la hora de quererse
invertir o negociar un instrumento financiero y al momento de aplicar
políticas monetarias.

\par

\textbf{rendimientos spot definir}

\par

Se procede a explicar qué es la curva de rendimiento, según Boudreault y
Reanud (2019) la curva de rendimiento también llamada curva de
rendimientos spot. Esta curva muestra la relación que hay entre las tasa
de rendimiento spot y el tiempo de maduración. Así esta curva muestra
los rendimientos que se obtienen de mantener una cierta cantidad de
dinero durante un tiempo determinado. Se puede determinar el rendimiento
mediante \[
y=\left(\frac{F}{B_0}\right)^{\frac{1}{T}}-1
\] Donde

\begin{table}[H]
\begin{tabular}{c|c}
$y$ & Tasa de rendimiento\\
$T$ & Tiempo para la maduración\\
$B_0$ & Costo de la inversión o precio del bono\\
$F$ & Monto a recibir al final del periodo
\end{tabular}
\end{table}
\par

Sin embargo a pesar de que esta herramienta es ampliamente utilizada,
tiene una limitante intertemporal. Como lo expone Camilo (2008) esta se
construye a través de una serie de tasas (precios) de instrumentos
financieros discontinuos en el tiempo. Esto implica que entonces lo se
ubica a partir de las tasas de rendimiemto es una serie de puntos que no
reflejan de forma continua las tasas de rendimiento en el mercado
financiero de acuerdo a su tiempo de maduración.

\par

Sin embargo hay agentes los cuales no están dispuestos a invertir o
prestar los montos al tiempo establecido por agentes como los gobiernos
nacionales u otras empresas. Es por esta razón que sería de utilidad
encontrar una curva suave la cual sea capaz de proyectar las tasas de
rendimiento en distintos momentos del tiempo, pero basado en lo que
anteriormente han establecido agentes como los gobiernos nacionales.

\par

Para lograr obtener una curva suave la cual se aproxime a las tasas de
rendimiento de agentes como los gobiernos, se emplean distintos métodos
entre ellos los métodos paramétricos que están basados en la
construcción de curvas a partir de modelos paramétricos. (Choudhry,
2010)

\par

Entre los modelos paramétricos se encuentra el de Nelson-Siegel. El cual
según Matteson (2015) describe las tasas forward con la siguiente curva
\[
r(t,\theta)=\theta_0+(\theta_1+\theta_2t)\exp(-\theta_3 t)
\] A partir de esta se puede obtener que la curva de rendimiento
continua se puede obtener haciendo \[
\begin{split}
y(t,\theta)&=t^{-1}\int_{0}^{t}r(x,\theta)dx\\
&=\theta_0+\left(\theta_1+\frac{\theta_2}{\theta_3}\right)\frac{1-\exp(-\theta_3 t)}{\theta_3 t}-\frac{\theta_2}{\theta_3}\exp(-\theta t)
\end{split}
\] Tal y como lo citan Hladíková y Radová (2012) este modelo tiene una
interpretación económica interesante para los parámetros. Tomando la
curva que describe las tasas forward Primero \[
\lim_{t\to\infty}r(t,\theta)=\theta_{0}\qquad \lim_{t\to 0}r(t,\theta)=\theta_0+\theta_1
\]

\begin{itemize}
\item Entonces $\beta_0>0$ representa la asintota horizontal de la curva 
\end{itemize}

\chapter{Descripción de los datos 5}

\par

Profesora, estuvimos investigado bases de datos donde nos dieran como
insumo el precio de los bonos, nos encontramos con varios inconvenientes
de las misma:

\par

\begin{enumerate}
\def\labelenumi{\arabic{enumi}.}
\tightlist
\item
  El registro de datos es muy corto, aproximadamente 1 mes como mucho,
  más que todo datos eran de un día especifico y no guardan registro de
  los anteriores.

  \par

  \begin{enumerate}
  \def\labelenumii{\arabic{enumii}.}
  \setcounter{enumii}{1}
  \tightlist
  \item
    En Bonos cero cupones, para calcular las tasas, el plazo de
    maduración máximo fue a 10 años, posterior a 10 años tienen cupón y
    no se puede hacer bootstrapping.

    \par

    \begin{enumerate}
    \def\labelenumiii{\arabic{enumiii}.}
    \setcounter{enumiii}{2}
    \tightlist
    \item
      No se tiene presente la tasa de descuento para estos bonos de más
      de 10 años, por lo que no se puede traer a valor presente, para
      calcular su respectiva tasa de pendiente

      \par

      \begin{enumerate}
      \def\labelenumiv{\arabic{enumiv}.}
      \setcounter{enumiv}{3}
      \tightlist
      \item
        Estos inconvenientes se mantenían en todas las bases de datos
        que hemos encontrado, las cuales han sido muy pocas por el
        siguiente punto.

        \par

        \begin{enumerate}
        \def\labelenumv{\arabic{enumv}.}
        \tightlist
        \item
          En sus mayorías las bases de datos con una cantidad aceptable
          de observaciones se encuentran las tasas de rendimiento (spot)

          \par

          Tal vez nos podría dar alguna indicación con respecto a estas
          bases de datos o sugerir una en específico, para así poder
          avanzar en el desarrollo del proyecto o en su defecto
          trabajarlo solo con las bases de datos que contienen a las
          tasas de rendimientos, como la que se está trabajando en este
          momento.

          \par

          Por lo anterior le presentamos dos ideas para desarrollar el
          trabajo, la primera partiendo de los precios de los bonos con
          una base de datos que a nuestro parecer no es la mejor, pues
          no cuenta con amplitud de observaciones ni rendimientos a
          largo plazo y la segunda con una base de datos que cuenta con
          las tasas de rendimiento únicamente, pero con grandes
          cantidades de observaciones y amplitud de fechas de
          maduración.
        \end{enumerate}
      \end{enumerate}
    \end{enumerate}
  \end{enumerate}
\end{enumerate}

\section{Opción 1}

\section{Opción 2}

Grafico de todos los datos:

\begin{Shaded}
\begin{Highlighting}[]
\NormalTok{Datosgra10Y}\OtherTok{\textless{}{-}}\FunctionTok{gather}\NormalTok{(datos10Y, }\StringTok{"Fecha de maduración"}\NormalTok{, Obtenido, }\FunctionTok{names}\NormalTok{(datos10Y[}\DecValTok{2}\NormalTok{])}\SpecialCharTok{:}\FunctionTok{names}\NormalTok{(datos10Y[}\FunctionTok{length}\NormalTok{(datos10Y)]))}

\FunctionTok{ggplot}\NormalTok{( }\AttributeTok{data =}\NormalTok{Datosgra10Y, }\AttributeTok{mapping =} \FunctionTok{aes}\NormalTok{(}\AttributeTok{x =}\NormalTok{ Date, }\AttributeTok{y =}\NormalTok{ Obtenido, }\AttributeTok{group=}\StringTok{\textasciigrave{}}\AttributeTok{Fecha de maduración}\StringTok{\textasciigrave{}}\NormalTok{, }\AttributeTok{color=}\StringTok{\textasciigrave{}}\AttributeTok{Fecha de maduración}\StringTok{\textasciigrave{}}\NormalTok{)) }\SpecialCharTok{+} 
    \FunctionTok{geom\_line}\NormalTok{()}\SpecialCharTok{+}
    \FunctionTok{scale\_color\_discrete}\NormalTok{(}\AttributeTok{labels=}\FunctionTok{c}\NormalTok{(}\StringTok{"1 mes"}\NormalTok{, }\StringTok{"6 meses"}\NormalTok{, }\StringTok{"1 año"}\NormalTok{, }\StringTok{"2 años"}\NormalTok{, }\StringTok{"3 años"}\NormalTok{,}\StringTok{"5 años"}\NormalTok{,}\StringTok{"7 años"}\NormalTok{, }\StringTok{"10 años"}\NormalTok{,}\StringTok{"20 años"}\NormalTok{,}\StringTok{"30 años"}\NormalTok{))}\SpecialCharTok{+}
      \FunctionTok{labs}\NormalTok{(}
      \AttributeTok{y =} \StringTok{"Tasa de rendimiento en \%"}\NormalTok{,}
      \AttributeTok{x =} \StringTok{"Tiempo"}\NormalTok{,}
      \AttributeTok{title =} \StringTok{"Comportamiento de las tasas spot"}
\NormalTok{      ) }\SpecialCharTok{+}
    \FunctionTok{theme}\NormalTok{(}\AttributeTok{legend.position=}\StringTok{"bottom"}\NormalTok{)}\SpecialCharTok{+}
    \FunctionTok{theme}\NormalTok{(}\AttributeTok{plot.title =} \FunctionTok{element\_text}\NormalTok{(}\AttributeTok{hjust =}\NormalTok{ .}\DecValTok{5}\NormalTok{))}\CommentTok{\# colocamos el titulo en el centro}
\end{Highlighting}
\end{Shaded}

\includegraphics{LABORATORIO2_files/figure-latex/unnamed-chunk-3-1.pdf}

Promedio y Desviación de los 3

\begin{Shaded}
\begin{Highlighting}[]
\CommentTok{\# ggplot( data= promedios10Y , mapping = aes(x = Maduration, y = Rendimiento)) + }
\CommentTok{\#     geom\_point()+}
\CommentTok{\#     geom\_point(data= promedios10Y, mapping = aes(x = Maduration, y =Varianza))+}
\CommentTok{\#   xlab(\textquotesingle{}Fecha de maduración\textquotesingle{}) + }
\CommentTok{\#   ylab(\textquotesingle{}Tasas spot\textquotesingle{}) +}
\CommentTok{\#   ggtitle(\textquotesingle{}Distribución de las tasa spot\textquotesingle{}) + }
\CommentTok{\#     theme(legend.position="bottom")+}
\CommentTok{\#     theme(plot.title = element\_text(hjust = .5))\# colocamos el titulo en el centro}
\end{Highlighting}
\end{Shaded}

\begin{Shaded}
\begin{Highlighting}[]
\CommentTok{\# ggplot(Datosgra10Y) +}
\CommentTok{\#   geom\_density(aes(x = Obtenido, fill = \textasciigrave{}Fecha de maduración\textasciigrave{}), position = \textquotesingle{}stack\textquotesingle{}) +}
\CommentTok{\# }
\CommentTok{\#   xlab(\textquotesingle{}Fecha de maduración\textquotesingle{}) +}
\CommentTok{\#   ylab(\textquotesingle{}Tasas spot\textquotesingle{}) +}
\CommentTok{\#   ggtitle(\textquotesingle{}Distribución de las tasa spot\textquotesingle{}) +}
\CommentTok{\#   theme\_minimal() +}
\CommentTok{\#   theme(legend.position="none")}
\end{Highlighting}
\end{Shaded}

\begin{Shaded}
\begin{Highlighting}[]
\FunctionTok{ggplot}\NormalTok{(}\AttributeTok{data =}\NormalTok{ Datosgra10Y, }\FunctionTok{aes}\NormalTok{(}\AttributeTok{x =}\StringTok{\textasciigrave{}}\AttributeTok{Fecha de maduración}\StringTok{\textasciigrave{}}\NormalTok{ , }\AttributeTok{y =}\NormalTok{ Obtenido, }\AttributeTok{group=}\StringTok{\textasciigrave{}}\AttributeTok{Fecha de maduración}\StringTok{\textasciigrave{}}\NormalTok{)) }\SpecialCharTok{+} \FunctionTok{geom\_boxplot}\NormalTok{(}\FunctionTok{aes}\NormalTok{(}\AttributeTok{color =} \StringTok{\textasciigrave{}}\AttributeTok{Fecha de maduración}\StringTok{\textasciigrave{}}\NormalTok{), }\AttributeTok{alpha =} \FloatTok{0.7}\NormalTok{) }\SpecialCharTok{+} 
  \FunctionTok{geom\_jitter}\NormalTok{(}\FunctionTok{aes}\NormalTok{(}\AttributeTok{color =} \StringTok{\textasciigrave{}}\AttributeTok{Fecha de maduración}\StringTok{\textasciigrave{}}\NormalTok{), }\AttributeTok{size =} \DecValTok{1}\NormalTok{, }\AttributeTok{alpha =} \FloatTok{0.02}\NormalTok{)}\SpecialCharTok{+}
  \CommentTok{\#geom\_violin(aes(fill = \textasciigrave{}Fecha de maduración\textasciigrave{}), color = \textquotesingle{}black\textquotesingle{}, alpha = 0.8)+}
  \FunctionTok{scale\_color\_discrete}\NormalTok{(}\AttributeTok{labels=}\FunctionTok{c}\NormalTok{(}\StringTok{"1 mes"}\NormalTok{, }\StringTok{"6 meses"}\NormalTok{, }\StringTok{"1 año"}\NormalTok{, }\StringTok{"2 años"}\NormalTok{, }\StringTok{"3 años"}\NormalTok{,}\StringTok{"5 años"}\NormalTok{,}\StringTok{"7 años"}\NormalTok{, }\StringTok{"10 años"}\NormalTok{,}\StringTok{"20 años"}\NormalTok{,}\StringTok{"30 años"}\NormalTok{))}\SpecialCharTok{+}
  \FunctionTok{xlab}\NormalTok{(}\StringTok{\textquotesingle{}Fecha de maduración\textquotesingle{}}\NormalTok{) }\SpecialCharTok{+} 
  \FunctionTok{ylab}\NormalTok{(}\StringTok{\textquotesingle{}Tasas spot en \%\textquotesingle{}}\NormalTok{) }\SpecialCharTok{+}
  \FunctionTok{ggtitle}\NormalTok{(}\StringTok{\textquotesingle{}Distribución de las tasa spot\textquotesingle{}}\NormalTok{) }\SpecialCharTok{+} 
  \FunctionTok{theme\_minimal}\NormalTok{()}
\end{Highlighting}
\end{Shaded}

\includegraphics{LABORATORIO2_files/figure-latex/unnamed-chunk-6-1.pdf}

Desviación estanda

\begin{Shaded}
\begin{Highlighting}[]
\NormalTok{Datosgra1Y}\OtherTok{\textless{}{-}}\FunctionTok{gather}\NormalTok{(datos1Y, }\StringTok{"Fecha de maduración"}\NormalTok{, Obtenido, }\FunctionTok{names}\NormalTok{(datos1Y[}\DecValTok{2}\NormalTok{])}\SpecialCharTok{:}\FunctionTok{names}\NormalTok{(datos1Y[}\FunctionTok{length}\NormalTok{(datos1Y)]))}

\FunctionTok{ggplot}\NormalTok{(}\AttributeTok{data =}\NormalTok{ Datosgra1Y, }\FunctionTok{aes}\NormalTok{(}\AttributeTok{x =}\StringTok{\textasciigrave{}}\AttributeTok{Fecha de maduración}\StringTok{\textasciigrave{}}\NormalTok{ , }\AttributeTok{y =}\NormalTok{ Obtenido, }\AttributeTok{group=}\StringTok{\textasciigrave{}}\AttributeTok{Fecha de maduración}\StringTok{\textasciigrave{}}\NormalTok{)) }\SpecialCharTok{+} \FunctionTok{geom\_boxplot}\NormalTok{(}\FunctionTok{aes}\NormalTok{(}\AttributeTok{color =} \StringTok{\textasciigrave{}}\AttributeTok{Fecha de maduración}\StringTok{\textasciigrave{}}\NormalTok{), }\AttributeTok{alpha =} \FloatTok{0.7}\NormalTok{) }\SpecialCharTok{+} 
  \FunctionTok{geom\_jitter}\NormalTok{(}\FunctionTok{aes}\NormalTok{(}\AttributeTok{color =} \StringTok{\textasciigrave{}}\AttributeTok{Fecha de maduración}\StringTok{\textasciigrave{}}\NormalTok{), }\AttributeTok{size =} \DecValTok{1}\NormalTok{, }\AttributeTok{alpha =} \FloatTok{0.04}\NormalTok{)}\SpecialCharTok{+}
  \CommentTok{\#geom\_violin(aes(fill = \textasciigrave{}Fecha de maduración\textasciigrave{}), color = \textquotesingle{}black\textquotesingle{}, alpha = 0.8)+}
  \FunctionTok{scale\_color\_discrete}\NormalTok{(}\AttributeTok{labels=}\FunctionTok{c}\NormalTok{(}\StringTok{"1 mes"}\NormalTok{, }\StringTok{"6 meses"}\NormalTok{, }\StringTok{"1 año"}\NormalTok{, }\StringTok{"2 años"}\NormalTok{, }\StringTok{"3 años"}\NormalTok{,}\StringTok{"5 años"}\NormalTok{,}\StringTok{"7 años"}\NormalTok{, }\StringTok{"10 años"}\NormalTok{,}\StringTok{"20 años"}\NormalTok{,}\StringTok{"30 años"}\NormalTok{))}\SpecialCharTok{+}
  \FunctionTok{xlab}\NormalTok{(}\StringTok{\textquotesingle{}Fecha de maduración\textquotesingle{}}\NormalTok{) }\SpecialCharTok{+} 
  \FunctionTok{ylab}\NormalTok{(}\StringTok{\textquotesingle{}Tasas spot en \%\textquotesingle{}}\NormalTok{) }\SpecialCharTok{+}
  \FunctionTok{ggtitle}\NormalTok{(}\StringTok{\textquotesingle{}Distribución de las tasa spot\textquotesingle{}}\NormalTok{) }\SpecialCharTok{+} 
  \FunctionTok{theme\_minimal}\NormalTok{()}
\end{Highlighting}
\end{Shaded}

\includegraphics{LABORATORIO2_files/figure-latex/unnamed-chunk-7-1.pdf}

\begin{Shaded}
\begin{Highlighting}[]
\NormalTok{Datosgra1Mo}\OtherTok{\textless{}{-}}\FunctionTok{gather}\NormalTok{(datos1Mo, }\StringTok{"Fecha de maduración"}\NormalTok{, Obtenido, }\FunctionTok{names}\NormalTok{(datos1Mo[}\DecValTok{2}\NormalTok{])}\SpecialCharTok{:}\FunctionTok{names}\NormalTok{(datos1Mo[}\FunctionTok{length}\NormalTok{(datos1Mo)]))}

\FunctionTok{ggplot}\NormalTok{(}\AttributeTok{data =}\NormalTok{ Datosgra1Mo, }\FunctionTok{aes}\NormalTok{(}\AttributeTok{x =}\StringTok{\textasciigrave{}}\AttributeTok{Fecha de maduración}\StringTok{\textasciigrave{}}\NormalTok{ , }\AttributeTok{y =}\NormalTok{ Obtenido, }\AttributeTok{group=}\StringTok{\textasciigrave{}}\AttributeTok{Fecha de maduración}\StringTok{\textasciigrave{}}\NormalTok{)) }\SpecialCharTok{+} \FunctionTok{geom\_boxplot}\NormalTok{(}\FunctionTok{aes}\NormalTok{(}\AttributeTok{color =} \StringTok{\textasciigrave{}}\AttributeTok{Fecha de maduración}\StringTok{\textasciigrave{}}\NormalTok{), }\AttributeTok{alpha =} \FloatTok{0.7}\NormalTok{) }\SpecialCharTok{+} 
  \FunctionTok{geom\_jitter}\NormalTok{(}\FunctionTok{aes}\NormalTok{(}\AttributeTok{color =} \StringTok{\textasciigrave{}}\AttributeTok{Fecha de maduración}\StringTok{\textasciigrave{}}\NormalTok{), }\AttributeTok{size =} \DecValTok{1}\NormalTok{, }\AttributeTok{alpha =} \FloatTok{0.02}\NormalTok{)}\SpecialCharTok{+}
  \FunctionTok{scale\_color\_discrete}\NormalTok{(}\AttributeTok{labels=}\FunctionTok{c}\NormalTok{(}\StringTok{"1 mes"}\NormalTok{, }\StringTok{"6 meses"}\NormalTok{, }\StringTok{"1 año"}\NormalTok{, }\StringTok{"2 año"}\NormalTok{, }\StringTok{"3 año"}\NormalTok{,}\StringTok{"5 año"}\NormalTok{,}\StringTok{"7 año"}\NormalTok{, }\StringTok{"10 año"}\NormalTok{,}\StringTok{"20 año"}\NormalTok{,}\StringTok{"30 año"}\NormalTok{))}\SpecialCharTok{+}
  \FunctionTok{xlab}\NormalTok{(}\StringTok{\textquotesingle{}Fecha de maduración\textquotesingle{}}\NormalTok{) }\SpecialCharTok{+} 
  \FunctionTok{ylab}\NormalTok{(}\StringTok{\textquotesingle{}Tasas spot en \%\textquotesingle{}}\NormalTok{) }\SpecialCharTok{+}
  \FunctionTok{ggtitle}\NormalTok{(}\StringTok{\textquotesingle{}Distribución de las tasa spot\textquotesingle{}}\NormalTok{) }\SpecialCharTok{+} 
  \FunctionTok{theme\_minimal}\NormalTok{()}
\end{Highlighting}
\end{Shaded}

\includegraphics{LABORATORIO2_files/figure-latex/unnamed-chunk-8-1.pdf}

Descripcion de lo de la pandemia

\begin{table}[H]

\caption{\label{tab:unnamed-chunk-9}Ejemplo de formato para tabla}
\centering
\begin{tabular}[t]{c}
\toprule
x\\
\midrule
1\\
2\\
3\\
4\\
5\\
\addlinespace
6\\
\bottomrule
\end{tabular}
\end{table}

\chapter{Metodología 1}

Cómo implementa la teoría expuesta en el marco teórico con sus datos,
colocaremos aquí los resultados que se vayan obteniendo, incluir al
menos un gráfico.

\includegraphics{LABORATORIO2_files/figure-latex/unnamed-chunk-10-1.pdf}
\includegraphics{LABORATORIO2_files/figure-latex/unnamed-chunk-10-2.pdf}
\includegraphics{LABORATORIO2_files/figure-latex/unnamed-chunk-10-3.pdf}

\chapter{Conclusiones y recomendaciones 1}

\chapter{Referencias bibliográficas}

\nocite{*} \printbibliography

\chapter{Anexos}

\backmatter
\end{document}
